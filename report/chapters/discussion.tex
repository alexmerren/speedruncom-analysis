\section{Discussion}


The analysis answers all the research questions stated earlier. Several different communities were found using the Louvain, CNM, and Infomap algorithms, with the Louvain algorithm performing the best on the user-game network with a modularity of 0.716. Most communities found using this method are extremely focused and most games within each share a distinct commonality. Two methods of recommendation systems were implemented, and both produced quantitatively and qualitatively good results. The collaborative filtering method produced only qualitative results, while the content filtering method produced a F1-score of 0.55 at 5 recommendations. The analysis of users of speedrun.com provided an insight into user behaviour and demographics, and the popularity of speedrun.com over time. The games of speedrun.com have a heavy imbalance with a few games being massively popular in most metrics. Finally, the collection process produced a lot of raw data from speedrun.com, and is available for distribution on GitHub.com.

\subsection{Exploratory Analysis}

Focusing on the exploratory analysis, it is not surprising that there is a heavy imbalance in the data with respects with to the number of users and number of runs for each game. There are approximately 32,000 games on speedrun.com, and most of them are extremely unpopular. Only a few games share most of the users, which can be defined as the most 'speedrunnable` games. These popular games are expected as they are either incredibly influential games such as Super Mario 64, or games with low barriers to entry and short speedrun times such as Seterra. These popular games remain similarly popular within the users that have only played a single game, indicating that most users will initially choose either a short or famous game to speedrun.


The number of runs and games played per user is not surprising. It was expected that most users have only played one game, but it is surprising that these users have more than one verified run. The expected user behaviour is that a user that tries speedrunning will only do so once, and then stop if they find it is too difficult or they are satisfied with their time. This is not the case, as users seem to submit a few times before achieving a desired time. The top users ranked by unique games played shows us that there are a few users that aim to play as many games as possible. There was an explicit competition between users named ``Gotta Run 'Em All'', with the explicit objective of playing as many games as possible. It is expected that the top users ranked by number of runs will be incredibly dedicated to a subset of games, submitting hundreds or thousands of runs for a single game. This is repeated for possibly all the games they play, as they have on average a low z-score of unique games played.


The popularity of the speedrun.com platform is completely unpredictable. It does have a slow increase in the number of new users joining the platform, but the increase from January 2020 to January 2021 is much higher than expected. The short-term trends within each year is correlated with the dates of Games Done Quick (GDQ) events, and the rapid increase in popularity in 2020 is most likely due to the COVID-19 pandemic. A study focusing on the amount of time spent playing video games both pre and post-pandemic shows that most respondents played far more video games post-pandemic \cite{gamesinpandemic}. This increase in the frequency of playing videogames would inevitably lead some to explore other ways of playing. The sharpest increase of new users occurs in October 2020. Unfortunately, no single event could be found to explain this increase. 


Speedrunning is mainly a phenomenon in the United States. No other demographic (apart from users that failed to submit their location) approaches the United States in size. It contains approximately 30\% of all users, and possibly more considering a portion of the `None` demographic are most likely from the US, but did not fill in their location. The other top demographics share many similarities, so finding a single contributing factor is incredibly difficult.


\subsection{Community Detection}

Although similar methods were applied to the user-game and the game-game graph, the results are incredibly different. The Louvain algorithm on the user-game graph had a 78.75\% increase in modularity over the game-game graph. This suggests that the communities on the user-game graph are well-defined compare to the game-game graph. This is true empirically, as the communities in the user-game graph tend to be more focused and meaningful. 


Communities zero and one both contain mostly Nintendo games, but are separated by the original platform of release. This separation suggests that users choose games based on the platform of release, or how old a videogame is. Community two contains games mostly for web or mobile, but investigating closely, it seems that these games have extremely low barriers for entry. This could mean that games are free-to-play, or only requiring a web-browser to play. Some communities are characterised by genre of games, such as community five containing platformers such as Celeste, community seven containing first-person perspective games like Half-Life, and community zero containing open-world games such as Grand Theft Auto 5. This suggests that users pick games based on the genre or underlying mechanics of the games. We also see communites based on a single game franchise such as communities six and 18. These communities are focused on Minecraft or Sonic games respectively, indicating that users will explore the same game franchises and the different speedruns they offer. 


The demographics within the communities does not change often. It was expected that some communities could contain users sharing a common language, or playing games from a certain region. Some communities do have these properties, but they are not large enough to characterise user behaviour as a collective.


The centrality analysis of the user-game and game-game graphs give some unexpected results. The degree centrality of games provides no interesting results, with the most popular games having the highest in-degree and out-degree. The Hubs and Authorities (HITS) centrality presents Authority nodes that tend to be those with high degree centrality, and Hub nodes that tend to be web or mobile games. PageRank offers the same results as the degree centrality. The betweenness centrality offers the most interesting results, with a higher betweenness centrality acting as bridges between communities. The most popular games within each community are usually those with the highest betweenness centrality. For example, community zero's highest betweenness centrality is Super Mario 64. Similarly, The Legend of Zelda: A Link to the Past has the highest betweenness centrality in community one. This trend is repeated through most of the communities, but notably missing in community two, which contains mostly web or mobile games. The highest betweenness centrality belongs to the Bee Movie Game (DS), which is unrelated to the other games in its community. This cannot be explained currently. These highest betweenness centrality games suggests that users explore new kinds of game or communities by playing the most popular game of that kind. The users then either explore this new community or return to their original preferences.

\subsection{Game Recommendation System}

Both methods of game recommendation systems produced anecdotally good recommendations. The recommendations were unexpectedly accurate, and were good at recommending games from the same game franchises. This suggests that users tend to play games within game franchises, and not random samples of the available games. Likewise, games with similar in-game mechanics or genre are also recommended, so users are likely to play these games together as well. 


The collaborative filtering method produced empirically more diverse recommendations. This may be advantageous when considering the cold-start problem of recommendation systems. The collaborative filtering method also takes users as input, rather than an individual game. This may be preferred when implementing and deploying a real-world game recommendation system, using only one request compared to several for each game a user plays, and ranking after considering their similarities.

\subsection{Implications}

Recalling the research questions stated earlier, it has been proven that there are communities of users of speedrun.com whose videogame preferences align with each other. Furthermore, these similarity of these users are reinforced with the accuracy of a game recommendation system with a reasonable accuracy and diversity of recommendations. This project provides concrete communities of users, and theories of the commonality between the users and games in these communities. Likewise, it analyses the users and games of speedrun.com and describes why they choose the games they play.