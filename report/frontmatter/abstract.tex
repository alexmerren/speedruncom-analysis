%!TEX root = ../main.tex
\begin{abstract}
The internet has allowed the transformation of real-life social circles to online social platforms, which aim to connect users similar to real-life interactions, but at a grand scale. This scale allows people to connect on previously uncommon behaviour. A recent phenomenon that can studied from this perspective is speedrunning: the practice of completing a videogame as quick as possible without cheating. Few studies quantitatively assess the act of speedrunning and the community surrounding the practice. We introduce a first perspective on investigating and interrogating the behaviour of users on speedrun.com. This project aims to determine communities of users on speedrun.com, and to create a game recommendation system using behavioural data from speedrun.com. We start by investigating community detection methods on a user-user and a user-game network, and determine the similarities between users and games within these communities. Second, this project implements content-based and collaborative filtering on users' game choice to identify trends and patterns in user behaviour. This project produces clusters of users and games, and the similarity between these clusters. Likewise, we create a game recommendation system, and a novel dataset of users and games from speedrun.com. This project finds users usually speedrun games to based on their platform of release, game genre, game mechanics, and specific game franchise. Similarly, accurate game recommendations can be made using these same criteria. This study provides an original application of community detection and recommendation system methods to the videogame community. 
\end{abstract}